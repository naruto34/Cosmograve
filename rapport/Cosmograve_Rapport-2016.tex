\documentclass[12pt, a4paper]{report}
\usepackage[utf8]{inputenc}
\usepackage[T1]{fontenc}
\usepackage{lmodern}
\usepackage{graphicx}
\usepackage{geometry}
\usepackage{lastpage}
\usepackage{newcent}
\usepackage[french]{babel}
\usepackage{fancyhdr}
\usepackage{colortbl}
\usepackage[usenames,dvipsnames,svgnames,table]{xcolor}
\pagestyle{fancy}

\geometry{hmargin=2.5cm,vmargin=1.5cm}
\renewcommand\headrulewidth{0pt}
\fancyhead[L]{}
\fancyhead[R]{}
\fancyfoot[C]{}
\fancyfoot[R]{Rapport\_Cosmograve\_2016-05-03 \textbf{Page \thepage/\pageref{LastPage}}}

\begin{document}
\begin{figure}[ht]
    \begin{minipage}[c]{0\linewidth}
        \centering
        \includegraphics[scale=0.2]{logo_lupm.png}
    \end{minipage}
    \hfill%
    \begin{minipage}[c]{0.36\linewidth}
        \centering
        \includegraphics[scale=0.4]{logo_um.png}
    \end{minipage}
\end{figure}

\title{rapport}
\author{GOURDIN Anthony, Mahuta Tamatoa, Hamza Alhousseini, Moussa-oumar Sy}

\begin{sloppypar}
\noindent {\huge\textbf{Cosmograve 2016}} \\
\end{sloppypar}
\bigskip
\noindent {\Large Rapport - Projet tuteuré de 1er année de Master\\Physique Numérique}

\vspace*{4cm}
\begin{center}
\begin{tabular}{|l|p{7cm}|}
\arrayrulecolor{lightgray}
  \hline
   & Noms\rule[-6pt]{0pt}{20pt} \\
  \hline
  Rédigé par : & Anthony GOURDIN \rule[-6pt]{0pt}{20pt}\\ 
   & Mahuta Tamatoa \rule[-6pt]{0pt}{20pt}\\ 
   & Hamza Alhousseini \rule[-6pt]{0pt}{20pt}\\ 
  \hline
  & Mr Reboul Henri\rule[-6pt]{0pt}{20pt} \\
  Encadré par :  & Mme Mougenot Isabelle \rule[-6pt]{0pt}{20pt}\\
   & Mr Cordoni Jean-Pierre \rule[-6pt]{0pt}{20pt}\\
  \hline
  Responsable pédagogique : & Mr Cassagne David\rule[-6pt]{0pt}{20pt} \\
  \hline
\end{tabular}
\end{center}

\tableofcontents
\thispagestyle{fancy}

\chapter*{ Remerciements}

Nous tenons à remercier Mme Mougenot qui nous à aider dans les problèmes informatique rencontrés, ainsi que Mr Cordoni et Mr Reboul pour leur conseil sur la compréhension de la théorie. Nous tenons aussi à les remercier pour leur disponibilité et leurs conseils d'ordre technique, documentation ou esthétique.
\\
\\
Nous remercions aussi Mr Cassagne pour l'organisation de ce projet tuteuré qui nous a permis d'apprendre énormément sur l'organisation en équipe notamment.

\chapter*{ Introduction}
\section{ Contexte du projet}
Dans le cadre du Master Physique Numérique, nous avons effectué un projet tuteuré encadré par Mme Mougenot pour la partie informatique, Mr Cordoni et Mr Reboul pour la partie physique.
\\
\\
Le projet tuteuré permet au étudiant concerné l'acquisition de compétences indispensables dans le monde de l'entreprise : \\
\begin{itemize}
	\item travail en équipe,
	\item respect d'un cahier des charges,
	\item rédaction d'un rapport,
	\item présentation orale devant un jury,
	\item recherche bibliographique et veille technologie.
\end{itemize}
\vspace*{7mm}
La mise en pratique des compétences acquise ou la recherche de nouvelle compétence suivra 3 grands aspects : \\
\begin{itemize}
	\item Analyse,
	\item Conception,
	\item Réalisation.
\end{itemize}

\section{ Contexte historique}
Dans notre programme, nous utilisons le modèle de Friedmann-Lemaître qui est basé sur les équation de la relativité générale mise en forme pour un contexte d'un modèle cosmologique homogène et isotrope.\\ \\
Au début du XIXe siècle, on assiste à la fin de l'espace absolu de Isaac Newton; selon laquelle il exister un espace absolue rigide et immuable, qui permettait de déterminer le mouvement d'un corps par rapport à un référentiel absolu.\\ Il devait aussi exister un temps qui était absolu et universel\cite{Esslinger}.\\ \\
Au XXe siècle, avec une meilleur compréhension de l'électromagnétisme; les physiciens inventaire l’éther une substance immatériel qui servait de support au onde électromagnétique et qui ne devait pas être rigide au point de perturber le déplacement des planètes.\\ \\
L'observation du phénomène d'aberration par l'astronome James Bradley en 1725, prouvait donc que la Terre devait être en mouvement par rapport à l'éther.\\
Pour mesurer la vitesse de la Terre par rapport à l'éther, en 1887, Michelson grâce à son interféromètre démontrât que la la Terre ne se déplaçait pas par rapport à l'éther.\\ \\
L'expérience de Michelson et l'aberration sont en contradiction; la seule conclusion est que l'éther ne pouvait pas exister.\\ \\
En 1905, la naissance de la relativité restreinte qui montre l'interdépendance de l'espace et du temps; l'idée d'intégrer la gravitation y apparaît assez vite comme la prochaine étape.\\ \\
En 1915, la relativité générale pousse encore plus loin la remise en cause de la physique de Newton.

\section{ Cosmograve (contexte et historique)}
Cosmograve est un programme qui a été initialement conçu dans le cadre du projet tuteuré du Master Physique Numérique en 2009.\\ 
\\ \\
Ces un programme qui ce veux avant tout pédagogique et facilement diffusable auprès d'un jeune publique avisé.
\\ \\
Cosmograve est composé de 2 parties,\\ la première consiste à étudier l'évolution du facteur d'échelle réduit de l'univers, déterminer plusieurs paramètres comme son age;\\ la deuxième consiste en la simulation de la trajectoire d'une particule autour d'un corps massif.

\part{Général}
\chapter{Choix du langage}

\chapter{Présentation}

\part{Cosmologie}

\part{Gravitation}

\part{Conclusion et source}
\newpage
\chapter{Conclusion}

\listoffigures
\thispagestyle{fancy}

\bibliography{biblio.bib}

\part{Annexes}

\newpage

\section*{Résume}

\section*{Abstract}

\end{document}